\chapter{Zahlentheoretische Grundlagen}

Hoher Durchsatz und geringe Ressourcen sind in vielen Anwendungen die entscheidenden Entwurfsparameter des ECC-Prozessors \cite{Hossain2019}.
Da die Effizienz von ECC-Prozessoren hauptsächlich von modularen
arithmetischen Operationen wie modularer Addition, Subtraktion und Multiplikation abhängt, ist der effiziente Entwurf modularer
Arithmetik eine sehr anspruchsvolle Aufgabe für die
Implementierung eines Hochleistungs-ECC-Prozessors \cite{Hossain2019}.\\ 

In diesem Kapitel werden die wichtigsten Arithmetikoperationen  über dem Primfeld $ \mathbf{ Z_p } $, die für die Kryptographie von Bedeutung sind, zusammen mit Beispielen vorgestellt und beschrieben. 

\section{Modulararithmetik}

Die modulare Arithmetik ist:
\begin{quote} \enquote{ein Prozess zum Reduzieren einer Zahl modulo einer
anderen Zahl, wobei dieser Prozess zahlreiche
Multi-Präzisions-Gleitkomma-Divisionsoperationen
umfassen kann \cite{patent}.
}\end{quote} 
Die modulare Artihmetik bietet endliche Strukturen, die alle üblichen arithmetischen Operationen der ganzen Zahlen aufweisen und die mit vorhandener Computerhardware problemlos implementiert werden können \cite{Contini2011}.
Eine wichtige Eigenschaft dieser Strukturen ist, dass sie durch Operationen wie Potenzieren zufällig permutiert zu sein scheinen, aber die Permutation kann oft leicht durch eine andere Potenz umgekehrt werden \cite{Contini2011}.
In entsprechend ausgewählten Fällen ermöglichen diese Vorgänge die Ver- und
Entschlüsselung oder die Generierung und Überprüfung von Signaturen
\cite{Contini2011}.\\

Die Berechnung hier erfolgt genauso wie bei der normalen Arithmetik
und der einzige Unterschied besteht darin, dass  alle
Operationen in der modularen Arithmetik in Bezug auf eine
positive ganze Zahl, das Modul, ausgeführt werden.
Ziel ist der kleinste positive Rest zu finden. \\

Im Allgemeinen lässt sich eine modulare Reduktion durch die folgende Gleichung definieren: \[ r = x \quad mod \quad p \quad = \quad x - \floor{\frac{x}{p}} \cdot p \], wobei x die zu reduzierende Anzahl modulo p ist, der gefundene Rest r, der im Bereich von [0, p - 1] liegt \cite{patent}. Daraus lässt sich die Äquivalenzrelation definieren und man sagt, dass $ r $ kongruent zu $ x $ modulo $ p $ ist.\\

Eine Zahl $ r $ ist \textbf{kongruent} bzw. \textbf{äquivalent} zu einer Zahl $ x $ modulo $ p $, ausgedrückt durch \( r \equiv x \pmod{p} \), falls p ein Teiler von ( r - x) ist. 
13 ?????? CITE???????


Dies bedeutet, dass r und x den gleichen Rest haben, wenn sie durch p geteilt werden und $ r = k $ $\cdot $ $ p + x $ mit  $ k \in $ $ \mathbf{Z} $.\\



In dieser Arbeit bildet  die Klasse \textit{BasicTheoreticMethod.java}, das Grundgerüst der Implementierung von elliptischen Kurven über das Primfeld $\mathbf{ Z_p} $. 
In dieser Klasse wurden folgende Funktionen implementiert:
\begin{itemize}
    \item \textit{modCalculation}, die den Rest aus der Division zweier ganzen Zahlen bestimmt; 
    \item \textit{isKongruent}, die prüft ob zweier Zahlen den selben Rest nach der Division;
    \item \textit{modAddition}
    \item \textit{modSubtraction}
    \item \textit{modMultiplikation}
    \item \textit{gcdExtended}
    \item \textit{hasInverse}
    \item \textit{multiplicativeInverse}
    \item \textit{modDivision}
    \item \textit{modExponentiation}
    \item \textit{phiFunction}
    \item \textit{chineseremainder}
\end{itemize}

Diese Klasse enthält unter anderen zahlentheoretische Methoden, die eine wichtige Bedeutung für die Kryptographie haben.
Die erste wichtige Methode ist die Methode der Berechnung von Modulo: $\textbf{ X $\pmod P $} $

\subsection{Berechnung von X mod Y}


Eine naive Methode wurde implementiert, in dem das Finden von $ r = x \pmod p $ zuerst
durch wiederholte Berechnung des Quotienten $ q = \frac{x}{p} $ und dann durch
wiederholtes Subtrahieren von x mit dem Ergebnis aus der Multiplikation von p mit q, bis das Ergebnis im Bereich von $ [0, p - 1] $ liegt.
Die Leistung der Modulo-Operation hängt vor allem hier von der Leistung des Divisionsalgorithmus ab. \\

Die naive Art eine ganzzahlige Division zu implementieren, besteht darin,
solange den Divisor p von dem Dividend x zu subtrahieren, bis der Dividend x negativ wird, und dabei den Quotienten q hoch zu zählen. Wenn der Dividend x negativ wird, wird er zum Divisor aufaddiert und der Quotient um eins verringert \cite{bericht}. 

Die implementierte Methode ist jedoch sehr langsam und kostspielig, denn wenn eine große durch eine kleine Zahl geteilt wird, muss man sehr oft Iterieren 
\cite{barett}. \\

Aber Methoden wie die \textbf{Barrett-Reduktion} können verwendet werden, um die Modulo-Operation zu optimieren (\cite{anoops}, \cite{hasenplaugh}).\\

Eine \textbf{Barrett-Reduktion} ist ein Verfahren zum Reduzieren einer Zahl modulo einer anderen Zahl, wobei die Division durch Multiplikationen und Verschiebungen ersetzt werden können, da die beiden letzteren Operationen viel billiger sind \cite{patent}.\\
Also der Quotient  $ q = \frac{x}{p} $ wird unter Verwendung kostengünstigerer Operationen mit Potenzen einer geeignet gewählten Basis $ b $ geschätzt. Für die Barett-Reduktion wird b häufig als Potenz von 2 gewählt werden.

Eine modulabhängige Größe $ m = \floor*{\frac{b^{2k}}{p}} $ muss berechnet werden, wodurch der Algorithmus für den Fall geeignet ist, dass viele Reduktionen mit einem einzigen Modul durchgeführt werden \cite{Hankerson}. Der Quotient $ q $ kann nun nur einmal für jedes Modul gleich dem Kehrwert von p berechnet werden \cite{barett}.

Dies erklärt sich aus der Tatsache, dass beispielsweise jede RSA-Verschlüsselung für eine Entität ein Reduktionsmodul für den öffentlichen Schlüssel dieser Entität erfordert \cite{menezes:1997}. \\

Das modulare Reduktionsverfahren umfasst \cite{patent}:
\begin{itemize}
       \item die Eingabe des zu reduzierenden Werts x und des Modulos p, wobei dies eine spezielle Form (z.B. NIST-Primzahl) hat,
       \item das Durchführen der modularen Reduktion von x mod p, wobei die modulare Reduktion umfasst:
       \begin{itemize}
         \item Berechnen eines genäherten Basisquotienten m,
         \item Berechnen einer Quotienten-Näherung q, 
         \item Berechnen eines genäherten Rests r, und
         \item Berechnen einer geschätzten, reduzierten Form des Wert x auf der Basis der Quotienten-Näherung und des genäherten Rests.
       \end{itemize}
\end{itemize}

 Die Tabelle \ref{tab3} veranschaulicht den Barett-Reduktion-Algortihmus. 



\begin{table}[!ht]
\centering
	\begin{tabular}{l}
		\toprule
		\textbf{Algorithmus: Modulo-Berechnung}\\
		\midrule
		Input: b > 3, p, $ k = \floor*{log_bp} + 1, 0 \leq x < b^{2k}, m = \floor*{\frac{b^{2k}}{p}} $\\
		Output: $ x  mod  p $ \\
		                                           \\
		                                           
		1. Berechne $ q = \floor*{ \floor*{\frac{x}{b^{k-1}}} \frac{m}{b^{k+1}} }\cdot p $;\\
		2. $ r = (x $ mod $ b^{k+1} ) - (q \cdot p $ mod $ b^{k+1})$;  \\
		3. If r < 0 then $ r = r + b^{k+1}$;\\ 
		4. While $ r \geq p $ do r = r - p; \\
		5. Return r. \\
	   \bottomrule
	\end{tabular}
	\caption{Barett-Reduktion\cite{nist}}
	\label{tab3}
\end{table}




Weitere grundlegenden und wesentlichen Operationen für die
Kryptographie sind die modulare Addition, modulare Subtraktion und modulare Multiplikation. 


\subsection{Modulare Addition}


Die modulare Addition über $ \mathbf{Z_p} $ kann mathematisch
geschrieben werden als:

\begin{ceqn}
\begin{align*}
    x + y \mod p 
\end{align*}
\end{ceqn} wobei x und y die
gegebenen Zahlen und p eine Primzahl sind. In dieser Arbeit wurde für die Implementierung der modularen Addition zuerst $ x $ und $ y $ 
addiert, dann wird das Ergebnis modulo p berechnet, wenn dies außerhalb des Bereichs von [0, p-1] liegt, sonst wird die Summe direkt ausgegeben. \\

Seien zum Beispiel x = 17, y = 4, p = 5, dann ist
\begin{ceqn}

\begin{align*}
   x + y \pmod p = \quad 17 + 4 \pmod 5 = \quad 21 \equiv 1 \mod 5 \\
\end{align*}
\end{ceqn}
Ein kostengünstiger bzw. schnellerer Algorithmus wäre zuerst x und y binär darzustellen und in einem Array von t-Bits zu speichern.
Danach sind x und y wortweise bzw.
bitweise zu addieren und vom Ergebnis dann p zu subtrahieren, solange es p - 1 überschreitet. \\

Jede Wortaddition erzeugt zum Beispiel in einer 32-Bit Plattform-Architektur eine 32-Bit-Summe und eine 1-Bit-Übertragsziffer, die zur
nächsthöheren Summe addiert wird \cite{nist}. Die einfache Addition und die \textit{Addition mit Übertrag} können schnelle Einzeloperationen sein \cite{Hossain2019}. Die Tabellen \ref{tab1} und \ref{tab2} veranschaulichen die beiden Algorithmen. 

\begin{table}[!ht]
\centering
\subfloat[1.Tabelle]{	
	\begin{tabular}{l}
		\toprule
		\textbf{Klassischer Algorithmus: Modulare Addition}\\
		\midrule
		Input: Modulo p und Zahlen x, y  $\in $ [0, p - 1]  \\
		Output: $ r = x + y \pmod p $ \\
		                                           \\
		                                           
		1. Berechne r = x + y;\\
		2. If $ r \neq 0 $, then finde den Rest r = r mod p else r = 0;  \\
		3. return r.\\ 
	   \bottomrule
	\end{tabular}}
	\caption{Modulare Addition aus der Klasse \textit{BasicTheoreticMethods}}
	\label{tab1}
%\end{table}
\qquad\qquad%
%\begin{table}[b]
%\centering

\subfloat[2.Tabelle]{
	\begin{tabular}{l}
		\toprule
		\textbf{Algorithmus: Modulare Addition}\\
		\midrule
		Input: Modulo p und Zahlen x, y  $\in $ [0, p - 1]  \\
		$ t = \ceil*{m/32} $ und $ m = \ceil*{log_2p} $   \\
		Output: $ c = (x + y) mod p $ \\
		                                           \\
		                                           
		1. $ c_0  = x_0 + y_0 $;\\
		2. For i from 1 to t-1 do: $ c = Add_-With_-Carry(x_i, y_i) $;\\
		3. If the carry bit is set, then subtract p from $ c = (c_{{t_-} 1}, ..., c_2,c_1,c_0)$;\\
		4. If $ c \geq p $  then $ c = c - p $;\\
		5. Return c.\\
       \bottomrule
	\end{tabular}}
    \caption{Empfohlene modulare Addition von amerikanischen Standard NIST für eine 32-Bit Architektur-Plattform \cite{nist}.}	
	\label{tab2}
\end{table}


\subsection{Modulare Subtraktion}


Mathematisch kann die modulare Subtraktion geschrieben
werden als: 
\begin{ceqn}
\begin{align*}
                     (x - y) \mod p 
\end{align*}
\end{ceqn} 

wobei x und y die gegebenen Zahlen und p die Primzahl sind. 
Der einfachste Weg diese Operation durch zu führen, besteht darin, die beiden Zahlen x
und y zuerst zu subtrahieren und dann der kleinste positive Rest im Bereich von [0, p-1] zu berechnen, in dem das Ergebnis
der Subtraktion von (x - y) durch p geteilt wird. Der Algorithmus sieht genauso aus wie der Algorithmus der modularen Addition, wobei im ersten Schritt anstatt eine Addition, eine Subtraktion durchgeführt wird. \\

Seien zum Beispiel x = 15, y = 23, p = 5, dann ist
\begin{align*}
    (x - y) \mod p = \quad (15-23) \mod 5 = \quad -8 \equiv 2 \mod 5 \\
\end{align*}

In der effizienteren Art, Modulo-Subtraktion durchzuführen, wird das Übertragungsbit nicht mehr als \texttt{Carry-Bit}, sondern als \texttt{Borrow-Bit} (Ausleih-Bit) interpretiert \cite{nist}. Die Operation wird dann ähnlich wie die modulare Addition implementiert.  

\subsection{Modulare Multiplikation}


Die modulare Multiplikation ist eine der teuersten und zeitaufwändigsten
Operationen der Kryptographie über $ Z_p$. Aber um ein höheres, leistungsstarkes Kryptosystem zu entwickeln,
muss eine effiziente Implementierung der modularen Multiplikation erfolgen \cite{Hossain2019}. 
Mathematisch kann die modulare Multiplikation-Operation ausgedrückt werden als:

\begin{ceqn}

\begin{align*}
           x \cdot y \mod p
\end{align*} 

\end{ceqn}

Um Modulo-Multiplikationen durchzuführen, wurde in dieser Arbeit die Zahlen x und y einfach multipliziert und dann die ganzzahlige Division durch p zu verwendet, um den Rest zu erhalten.
Dies bedeutet auch, dass die Leistung des ganzen Algorithmus, genauso wie bei der modularen Addition und Subtraktion, von Modulo-Operationen bzw. von der Leistung des Divisionsalgorithmus abhängt. \\
Seien zum Beispiel: x = 35, y = 7, p = 5, dann gilt:
\begin{align*}
    x \cdot y \mod p = \quad 35 \cdot 7 \mod 5 = \quad 245 \equiv 0 \mod 5 \\
\end{align*}

Neben der Barett-Reduktion Methode gibt es auch eine andere Methode, die die
Modulo-Operationen viel schneller als die reguläre klassische Methode
durchführen kann: Die \textbf{Montgomery-Multiplikation}. \\

Die \textbf{Montgomery-Multiplikation} ist ein Verfahren zur modularen Multiplikation, bei der die traditionelle Divisons-Operation vermeidet wird und anschließend nur Multiplikationen, Additionen und Verschiebungen verwendet werden \cite{sahuMa}. Die Zahlen x und y werden binär dargestellt. 
Der modulare multiplikative Algorithmus ist in Tabelle \ref{tab4} angegeben.

\begin{table}[!ht]
\centering
	\begin{tabular}{l}
		\toprule
		\textbf{Algorithmus: Modulare Multiplikation}\\
		\midrule
		Input: p, x und y zwei positive k-Bit Ganzzahlen, $ x_i $, $ y_i $: i-te Bit in x und y\\
		Output: $ m = x \cdot y $ mod $ p $ \\
		                                           \\
		                                           
		1. m = 0;\\
		2. For i = 0 to k-1 \\
		   \quad2.1 \quad $ m = m + ( x \cdot y_i) $;\\
		   \quad2.2 \quad If $ ( m_0 = 1 ) $ then $ m = m/2 $ \quad else \quad $ m = (m + p)/2 $;\\ 
	    3. Return m. \\
	   \bottomrule
	\end{tabular}
	\caption{Montgomery-Multiplikation \cite{sahuMa}}
	\label{tab4}
\end{table}

Das Verfahren ist für eine einzelne modulare Multiplikation nicht effizient,
kann jedoch effektiv bei Berechnungen wie der modularen Potenzierung
verwendet werden, bei denen viele Multiplikationen für eine gegebene Eingabe durchgeführt werden \cite{Hankerson}. Im nächsten Unterabschnitt werden Methoden zur Berechnung der modularen Potenzierung erläutert.

\subsection{Modulare Exponentiation}

Das Problem der modularen Potenzierung lässt sich mathematisch definieren als:
\begin{ceqn}

\begin{align*}
   A = x^k \mod \quad p \quad mit \quad x \in \mathbf{Z_p} \quad und \quad 0 \leq k < p
\end{align*}

\end{ceqn}

Die modulare Potenzierung ist die kostspielige, aber
entscheidende und bedeutungsvolle Methode für viele kryptographische Protokolle. \\

Algorithmen zur schnellen modularen Potenzierung wie das Square-and-Multiply-Verfahren, basieren vor
allem auf der Auswertung der Binärdarstellung des Exponenten k \cite{langMEI}. 
Dort wird der Exponent k binär dargestellt, als k = \(\sum_{i=0}^{t} k_i \cdot 2^i\) mit $ k_i \in \{0, 1\}$ und die Auswertung kann dabei je nach Algorithmen entweder von links nach rechts oder von rechts nach links erfolgen \cite{langMEI}. \\

Bei einem k-Bit-Exponenten werden dann für die Potenzierung höchstens k Quadrate und k Multiplikationen benötigt \cite{langMEI}.
Selbst wenn sie effizient mit dem wiederholten Square-and-Multiply-Verfahren durchgeführt wird, erreicht sie eine Bit-Komplexität von $\mathcal{O}(\log{}n^3)$ \cite{menezes:1997}.  \\

Gegeben sei beispielsweise \(x^k = x^{13}\). Zu berechnen ist, \( 2^{13} \mod 7\).
Dann ist die binäre Darstellung von k = 1101. Der Exponent k lässt sich wie folgt auswerten:
\begin{ceqn}
  \[((((0) \cdot 2 + 1) \cdot 2 + 1) \cdot 2 + 0) \cdot 2 + 1   =   13 \]
\end{ceqn}

Daraus lässt sich $ x^{13} $ bzw. $ 2^{13}$ wie folgt aufteilen:\\
\begin{ceqn}
   \[ x^{13} = ((((x^0)^2 \cdot x^1)^2 \cdot x^1)^2 \cdot x^0)^2 \cdot x^1 \] \\
   \[ 2^{13} = ((((2^0)^2 \cdot 2^1)^2 \cdot 2^1)^2 \cdot 2^0)^2 \cdot 2^1 \] 
\end{ceqn}

Es ergibt sich, dass \( 2^{13} \equiv 2 \mod 7\).\\

Die schnelle modulare Potenz kann jedoch auch in einfacher Weise realisiert werden \cite{langME}. Tatsächlich, kann \( x^{13}\) auch dargestellt werden, als 

\begin{align*}
    x^{13} = x^{12} \cdot x \quad \text{und \( x^{12}\) weiterhin als} \quad x^{12} = (x^{6})^2.
\end{align*}

Ausgehend von dieser Beobachtung, ergibt sich die folgende Rekursionsformel \cite{langME}:

\begin{ceqn}
\begin{align*}
     x^k = \begin{cases}
     1    & \text{ falls } k = 0 \\
    \;\;x^{k-1}.x  & \text{ falls } k \quad ungerade \\
    (x^{k/2})^2     & \text{ falls } k \quad gerade \\
    \end{cases}
\end{align*}
\end{ceqn}

In dieser Arbeit wurde für die Implementierung der modularen Potenz, die Idee der Rekursionsformel verfolgt.
In dieser rekursive Implementierung wurde die Binärdarstellung des Exponenten k nicht explizit benötigt.\\

Der Algorithmus startet mit Basisfällen, in dem geprüft wird, ob die Zahlen x, k und p in den richtigen Bereichen liegen. 
A wird am Anfang auf 1 gesetzt (A = 1). Wenn k gleich null ist, wird A = 1 zurückgegeben. 
Wenn k größer null ist, wird in jedem Schritt geprüft, ob der Exponent k ungerade bzw. ob die aktuelle binäre Zahl eins ist. 
Wenn ja, wird das Ergebnis A mit x multipliziert und das Modulo berechnet; ansonsten erfolgt zuerst eine
Rechtsverschiebung (k wird halbiert) und danach wird x quadriert. Danach wird der Rest bestimmt, damit $x \in [0, p-1] $ bleibt.
Diese Schritte werden durchgeführt, solange der Exponent größer 0 ist. Der Algorithmus wird in der Tabelle \ref{tab5} dargestellt.
Also die Leistung dieser Methode hängt von der Leistung des Modulo-Algorithmus ab.

\begin{table}[!ht]
\centering
	\begin{tabular}{l}
		\toprule
		\textbf{Algorithmus: Modulare Potenz}\\
		\midrule
		Input: p, x, $ k $ $\in$ $\mathbf{Z} $ \\
		Output: $ x^k \mod $ p \\
		                                           \\
		                                           
		1. A = 1;\\
		2. If $ (k = 0)$ then return A;\\
		3. If $ (k < 0) $ then $ k = \lvert k \rvert $;\\
		   \quad x = ($x^{-1} \mod $ p) $ \mod $ p;\\
		4. x = x $ \mod $ p; \\
		5. while (k > 0) \\
		 \quad 5.1\quad If $ (k $ \& $ 1)$ = 1  then  A = A $ \cdot $ x $ mod $ p; \\
		 \quad 5.2 \quad k $ \gg $ 1; \\
		 \quad 5.3 \quad x = x $ \cdot $ x $ \mod $ p;\\
	    6. Return A. \\
	   \bottomrule
	\end{tabular}
	\caption{Modulare Potenz aus der Klasse \textit{BasicTheoreticMethods.java}}
	\label{tab5}
\end{table}

In den meisten wissenschaftlichen Artikeln, liegt $ k \in [0, p-1] $, aber in dieser Arbeit es wurde angenommen, dass k $ \in \mathbf{Z}$ liegt. 
Betrachtet wurde also die Fälle, wo der Exponent k auch negative
Werte haben kann, um möglichst alle Sicherheitslücken  auszuschließen. \\

In diesem Fall wird für die Berechnung der modularen
Potenz, der Absolutwert des Exponenten genommen, aber
vorher nimmt x den
resultierenden Wert aus der Berechnung der modularen multiplikative Inverse von x und p an.


\subsection{Modulare multiplikative Inverse}

Das modulare Inverse ist eine weitere wichtige Methode der Kryptographie. Wenn a eine Restklasse aus $\in \mathbf{Z_p} $  (geschrieben $[a]_p$) und p eine Primzahl ist, ist das
modulare Inverse $ a^{-1} $ eine ganze Zahl, die die Beziehung 
\begin{ceqn}
 \begin{align*}
     a \cdot a^{-1} \equiv 1 \mod p 
 \end{align*}
\end{ceqn} erfüllt.

Das modulare Inverse wird mit Hilfe des erweiterten euklidischen (EEA) bestimmt.
Dort sind die ganzen Zahlen x und y zu finden, für die die folgende Gleichung erfüllt ist: \\
$ ggT (a, p) = 1 = ax + py $. \\
Also muss der $ ggT (a, p) = 1 $ sein, da p prim ist. Wenn dies den Fall lässt sich feststellen, dass a und p \textbf{Koprime} sind.
\begin{align*}
    \textbf{Extended Euklidischer Algortihmus}
\end{align*}
Der extended Euklidische Algorithmus basiert auf dem euklidischen Algorithmus, der den ggT von zwei ganzen Zahlen durch wiederholte Anwendung des Divisionsalgorithmus ermittelt. Aber bei der Berechnung des ggT wird auch der Wert von x verfolgt. \\

Der erste Schritt des euklidischen Algorithmus besteht darin, die größere Ganzzahl durch die Kleinere zu teilen. Dann wird der Divisor wiederholt durch den Rest geteilt, bis der Rest 0 ist. Der ggT ist dann der letzte Rest ungleich Null in diesem Algorithmus. \\
Allerdings die Voraussetzung für die Bestimmung der Inverse, dass $ ggT(a, p) = 1 $ sein muss. 
Wenn dies der Fall ist, können durch Umkehren der Schritte im euklidischen Algorithmus die ganzen Zahlen x und y gefunden werden.

Dies kann erreicht werden, indem die Zahlen als Variablen behandelt werden, bis den Ausdruck \\
\(1 = x \cdot a + y \cdot p \) erhalten wird, der eine lineare Kombination unserer Anfangszahlen ist. \\
Daraus es folgt \( x \cdot a \equiv 1 \mod p \). 

Zum Beispiel, gesucht wird das Inverse von $[17]_{53}$. \\
EEA liefert:
\begin{ceqn}
\begin{align*}
              53 = 3 \cdot 17 + 2 \\
    \qquad    17 = 2 \cdot 2 + \boxed{1}   \\
    \qquad    2 = 2 \cdot 1 + 0
\end{align*}
\end{ceqn}
Es ergibt sich, dass ggT(17, 53) = 1. Es folgt: \\ 

\begin{ceqn}
\begin{align*}
              1 = 25 \cdot 17 + (-8) \cdot 53 \\
              1 \equiv 25 \cdot 17 \pmod 53
\end{align*}
\end{ceqn}
d.h. das Inverse von $[17]_{53}$ ist $[25]_{53}$. \\

Das modulare multiplikative Inverse lässt sich auch durch die modulare Potenz unter Verwendung des Satz von Euler berechnen. \\
\textbf{Satz von Euler}
Sei \(n \in \mathbb{N}\), die Menge der natürlichen Zahlen. Dann gilt
für alle a \(\in \mathbb{N}\), die teilerfremd ( $ ggT(a, n) = 1 $) zu n sind also \cite{langMIE}: \\

\begin{ceqn}
\begin{align*}
            a^{\varphi(n)} \equiv 1 \pmod n
\end{align*}
\end{ceqn}
mit $ \varphi(n)$, die Euler’sche $\varphi$-Funktion, die die Anzahl der Elemente der
reduzierten Menge der Reste modulo n bezeichnet \cite{damer}. \\

Nach dem Satz von Euler gilt für jedes \(a \in \mathbf{Z^*_n} \)\\

\begin{ceqn}
\begin{align*}
            a^{\varphi(n)} \mod n = 1 \\
    \qquad  a^{\varphi(n) - 1} \mod n = a^{-1}
\end{align*}
\end{ceqn}

Die Euler'sche $\varphi$-Funktion ist gleich der Anzahl der zu n
teilerfremden Zahlen zwischen 1 und n. Man schreibt \\

\begin{ceqn}
\begin{align*}
         \varphi(n) = |\{ 1 \leq a < n, mit \quad ggT(a, n) = 1 \}|
\end{align*}
\end{ceqn}

Für Primzahlen gilt \(\varphi(p) = p - 1 \). \\

Durch modulare Potenzierung lässt sich zwar das Inverse einer Restklasse leicht bestimmen, aber sie ist im Vergleich mit der Berechnung mit dem erweiterten euklidischen Algorithmus etwas langsamer. Außerdem muss der Wert der Variable \(\varphi(n)\) und damit die Kenntnis der Primfaktorzerlegung von n vorher bestimmt sein \cite{langMIE}.

Mit Hilfe des modularen multiplikativen Inverses lässt sich leicht die modulare Division zwischen 2 Zahlen bestimmen.


\subsection{Modulare Division}

Die modulare Division lässt sich mathematisch definieren als:
\begin{ceqn}
\begin{align*}
    x/y \mod p \quad = \quad x \cdot y^{-1} \mod p \\
\end{align*}
\end{ceqn}

Bei der Bestimmung der modularen Division wird zuerst $ y^{-1}$, das modulare Inverse von y über p berechnet. Danach wird die modulare Multiplikation von $ x \cdot y^{-1} $ durchgeführt.  

\section{Chinesischer Restsatz}

Eine weitere Methode, die eine wichtige Rolle in der
Entschlüsselung bzw. in der Generierung von Schlüsseln für die
Kryptographie spielt, ist der chinesische Restsatz.
Der chinesische Restsatz (engl. Chinese Remainder Theorem: CRT) besagt, dass wenn 2 Zahlen p und q Koprime
sind (also,wenn ggT(p,q) = 1), dann hat das Gleichungssystem:

\begin{ceqn}
\begin{align*}
      x \equiv a \mod p \\
      x \equiv b \mod q 
\end{align*}
\end{ceqn}
eine eindeutige Lösung für \(x \mod p \cdot q \). 

Also wenn p und q Koprime sind, existieren 2 Zahlen $ m_1 $ und $ m_2 $, so dass $ m_1 $ $\cdot$ p + $ m_2 $ $\cdot $ q = 1 gilt. Um $ m_1 $ und  $ m_2 $ zu bestimmen, wird der EEA verwendet und daraus ergibt sich, dass

\begin{ceqn}
   \begin{align*}
       x = a \cdot m_2 \cdot p + b \cdot m_1 \cdot q, \quad \text{für 2 Gleichungen ist.}
   \end{align*}
\end{ceqn}
Für beliege Gleichungen:
\begin{align}
     x = \sum_{i=1}^{k} a_i \cdot N_i \cdot R_i \mod N, \quad \text{mit k, die Anzahl der Gleichungen;}
     \label{crt}
\end{align}
\begin{align*}
     N_i = N/n_i \quad \text{ und} \quad R_i = N_i^{-1} \mod N.
\end{align*}

In der Tat ermöglicht dieser Satz, eine Nachricht zu entschlüsseln, in dem am Anfangs eine geheime Nachricht M in
beliebige Hälften (hier in 2 Hälften $ m_1, m_2 $) aufgeteilt wird und dann das Modulo jede diese Nachricht berechnet, wenn die Faktorisierung des
öffentlichen Schlüssels $ N = p \cdot q $ bekannt ist. Danach
werden diese Nachrichten in einer Zahl x wieder kombiniert sein und anschließend wird der private Schlüssel berechnet.


Gegeben sind zum Beispiel zwei Zahlen p = 5 und q = 7. Gesucht ist die Zahl x für die gilt:
\begin{ceqn}
\begin{align*}
      x \equiv 2 \mod 5 \\
      x \equiv 3 \mod 7 
\end{align*}
\end{ceqn}

Berechnung des ggT (5,7) mittels des EEA ergibt 1 = 3 $\cdot $ 5 - 2 $\cdot $ 7, also es existieren 2 Zahlen $ m_1, m_2 $, mit $ m_1 =
3$ und $ m_2 = -2 $. \\
Dann ist 
\(x = 2 \cdot -2 \cdot 7 + 3  \cdot 3 \cdot 5\) = \( -28 + 45  \equiv 17 \mod 35\). \\

Für diese Arbeit wurde für die Implementierung der Gleichung \ref{crt} betrachtet. Zwei Array-Listen wurden genutzt, um alle Reste $ a_i $ und alle Modulo-Werte $ N_i $ zu speichern. K wird als die Größe der Liste genommen und der Algorithmus wird nur laufen, für Listen gleicher Länge. Die Tabelle \ref{tab6} stellt den entsprechenden Algorithmus dar. 
\begin{table}[!ht]
\centering
	\begin{tabular}{l}
		\toprule
		\textbf{Algorithmus: Chinesischer Restsatz}\\
		\midrule
		Input:$ a = \{ a_1,..., a_k\}, n = \{n_1,..., n_k\} $, wobei alle $ n_i $ co-prime sind \\
		Output: $  x = \sum_{i=1}^{k} a_i \cdot N_i \cdot R_i \mod N $ \\
		                                           \\
		                                           
		1. x = 0;\\
		2. k = n.size();
		2. If $ (k > a.size())$ then k = a.size();\\
		3. For \(i = 0\) bis k-1 compute product of all moduli N =\( n_1*...*n_k \); \\
		4. For \(i = 0\) bis k-1 \\
		 \quad 4.1\quad compute $ N_i = N/n_i $;  \\
		 \quad 4.2 \quad compute Inverse $ \ R_i = N_i^{-1} \mod n_i \ $; \\
		 \quad 4.3 \quad x = \ x + $ a_i $ $\cdot$ $ R_i $ $\cdot $ $  N_i $ \;\\
	    5. Return \(x \mod N.\) \\
	   \bottomrule
	\end{tabular}
	\caption{Chinesischer Restsatz aus der Klasse \textit{BasicTheoreticMethods.java}}
	\label{tab6}
\end{table}

Die Gleichung aus \ref{crt} wird als Gauß-Algorithmus bezeichnet.
Die Berechnungen können in $\mathcal{O}(\log{}n^2)$ Bitoperationen
durchgeführt werden \cite{menezes:1997}. 
 



\section{Polynomarithmetik}