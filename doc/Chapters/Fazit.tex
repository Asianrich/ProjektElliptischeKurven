\chapter{Fazit}

\section*{Annick:}
Da ich noch keine Berührung mit endlichen Körpern und modularer Arithmetik hatte, war auch die Elliptische Kurve mathematisches Neuland für mich, sodass ich einen schweren Start in dieses Projekt hatte. Ich musste zuerst einiges Wissen recherchieren und nachlesen um dann meine Klassen implementieren zu können. Daher bin ich sehr zufrieden, dass das recht reibungslos geklappt hat und unsere Klassen und Methoden ohne größere Probleme miteinander kompatibel waren. 


\section*{Richard:}
Das Projekt ist außerordentlich interessant. Da ich schon ein Semester Kryptographie hatte, 
ist die Elliptische Kurve kein Fremdwort mehr. Die Probleme die anfänglich und während des Projektes auftraten, waren schlechtes Zeitmanagement und Kommunikationsprobleme.
Diese sind durch einige Wahlpflichtfächer mit großen Anforderungen und auch wegen der Pandemie, die ein regelmäßiges persönliches Treffen unmöglich machte, zu erklären.\\
\\
Dennoch haben wir das beste daraus gemacht. Die größte Schwierigkeit in diesem Projekt war der Anfang des Projektes.
\\
Alle mussten auf dem gleichen Stand sein, so dass keine neuen Probleme mit dem Thema entstehen.
Je mehr organisatorische Probleme gelöst wurden, desto weniger Konflikte entstanden.
\\
\\
In meinen Teil waren der Fermat-Test und die endlichen Körper relativ schnell gelöst.
Diese sind nicht so kompliziert wie man auf den ersten Blick denkt.\\
Lediglich war die Primzahlpotenz sehr komplex, denn zuerst hatte ich die falsche Quellen genommen und dadurch das Thema falsch verstanden.\\
Dies hat dazu geführt, das ich das Kapitel komplett neu machen musste, jedoch lernte ich schnell dazu und somit hatte ich eine neue Grundlage um das Kapitel aufzubauen.\\
\\
Zuerst musste man verstehen, das es nur Polynome berechnet werden und kein richtiger Wert erwartet wird. 
Natürlich wird man neugierig und will viel mehr machen, jedoch hat alles eine Zeitgrenze und es wurden nur die Grundlagen implementiert.\\
\\
Dieses Projekt hat mein persönliches Interesse an die Mathematik der Kryptographie geweckt. 

\section*{Hendrik:}
Ähnlich wie Richard hatte ich mich schon vor dem Projekt mit dem Thema Elliptische Kurven beschäftigt. Daher fiel es mir mehr oder weniger leicht die Implementierung der verschiedenen Darstellungen zu implementieren. Besonders die Mathematik hinter der Punktaddition und die daraus resultierenden SpeedUps faszinieren mich. Hinzu kommt das Phänomen, dass man mit jeder Iteration der Punktaddition wieder einen Punkt auf der Kurve trifft und dadurch einen sicheres Verfahren für einen Schlüsselaustausch bereitstellt.\\
Natürlich gab es auch kleinere Probleme während der Projektzeit, aber die sind für mich nebensächlich, da ich einen guten Gesamteindruck von unserer Teamarbeit habe. Schließlich haben wir alle gut zusammengearbeitet und können ein ordentliches Programm vorweisen.

