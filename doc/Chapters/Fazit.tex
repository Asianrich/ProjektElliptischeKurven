\chapter{Fazit}

Richard:\\
\\
Das Projekt ist außerordentlich interessant. Da ich schon 1 Semester Kryptographie hatte, 
ist die Elliptische Kurve kein Fremdwort. Die Probleme die anfänglich und während des Projektes waren, die etwas schlechte Zeitmanagement und Kommunikationsprobleme.
Dieser kam durch die einige Wahlfächer mit deren Anforderungen und auch wegen der Pandemie, kann man schlecht zusammentreffen.\\
\\
Dennoch haben wir den Mut zusammengefasst und haben das beste daraus gemacht. Die Schwierigkeit in diesem Projekt, war der Anfang des Projektes.
\\
Alle mussten auf dem gleichen Stand sein, so dass keine Probleme über das ganze Thema entstehen.
Je mehr organisatorische Probleme gelöst wurden, so kommt es zu wenigen Konflikten.
\\
\\
Zu meinen Teil Endlicher Körper, Fermat-Test, Primzahlpotenz, waren Fermat-Test und der Endliche Körper, relativ schnell gelöst.
Diese sind nicht so kompliziert wie man denkt.\\
Lediglich ware die Primzahlpotenz sehr komplex. Denn zuerst hatte ich die falsche Quellen genommen und dabei das Thema falsch verstanden.\\
Dies hat dazu geführt, das ich den Kapitel komplett neu machen musste. Jedoch wurde es schnell erlernt und somit hatte man eine neue Grundlage um das Kapitel aufzubauen.\\
\\
Zuerst musste man verstehen, das es nur Polynome berechnet werden und kein richtigen Wert erwartet. 
Natürlich wird man neugierig und will viel mehr machen, jedoch hat alles eine Zeitgrenze und es wurde nur die Grundlagen eingerichtet.\\
\\
Dieses Projekt hat mich persönlich mehr Interesse an die Mathematik der Kryptographie gegeben. 

