%********************************************************************
% Appendix
%*******************************************************
\chapter{Darstellung von Punkten}
Um Punkte auf einer elliptischen Kurve darzustellen oder anzugeben gibt es mehrere Möglichkeiten, die Unterschiedliche Vor- und Nachteile haben. Jede Darstellung wird daher auch anders miteinander verrechnet.
\subsection{Affine Darstellung}
Die affine Darstellung ist die gebräuchlichste Art und Weise einen Punkt und eine elliptische Kurve darzustellen, da es nur die x und y Koordinaten gibt.\\
In diesem Fall gibt es keine einheitliche Darstellung der Unendlichkeit und man muss prüfen, ob die Koordinaten auf der Kurve liegen.
\subsubsection{Addition}
Die Addition zweier Punkte bedeutet, dass man eine Gerade durch die Punkte zieht und diese Gerade schneidet die Kurve in einem dritten Punkt, dieser ist das Ergebnis.\\
Wenn man den gleichen Punkt mit sich selbst addiert, dann muss wie immer bei der Addition die Steigung berechnet werden. \[m = 3x^2+A / 2y\]
woraus sich wiederum die x Koordinate \[m^2-2x\] und y Koordinate berechnet wird \[y = m(x - x_0)+y\]\\
Bei zwei unterschiedlichen Punkten, wovon einer nicht im unendlichen Bereich liegt, wird die Steigung durch den Differenzquotienten berechnet: \[m = x_1-x_2 / y_1 - y_2\]
Nun braucht man nur noch einzusetzen \[x = m^2 -x1 - x2\] \[y = m(x-x_1)-y_1\] und erhält den neuen Punkt.\\
Ansonsten gilt \(P_1 + \infty = P_1\), wobei \infty ein Punkt außerhalb der Kurve ist.\\
Diese Addition ist für den Rechner bei großen Zahlen sehr aufwendig, da man dividieren muss und das die anspruchsvollste Rechenoperation ist. Man benötigt zwei Multiplikationen, eine Quadrierung und eine Division.
\subsubsection{Negation}
Die Negation wird durch einen Vorzeichenwechsel durchgeführt bei der y-Koordinate. Da die Zahl dann negativ ist, muss noch das positive Gegenstück im endlichen Zahlenraum gefunden werden, was durch die einfache Addition der Zahl p geht.
\subsection{Projektive Darstellung}
Bei der projektiven Darstellung kommt die z - Koordinate hinzu, sodass aus der zwei dimensionalen eine drei dimensionale Kurve entsteht. Die Umrechnung eines affinen Punktes in einen projektiven Punkt ist dabei einfach durch das Hinzufügen von z = 1 durchzuführen. Bei der umgekehrten Richtung muss man die x und y Koordinaten durch z teilen, woraus folgt, dass alle Punkte mit z = 0 unendlich sind, da sie nicht auf der Kurve liegen können. Zur Erinnerung, die Formel der Kurve ändert sich von \(y^2 = x^3 + Ax + B\) zu \(y^2z = x^3 + Axz^2 + Bz^3\)\\
Eine Punktaddition benötigt zwölf Multiplikationen und zwei Quadrierungen, während eine Punktverdopplung nur sieben Multiplikationen und fünf Quadrierungen braucht.
\\
\subsubsection{Addition}
Bei \(P1 \neq P2\) sieht die Addition wie folgt aus:
\[u = y_2z_1 - y_1z_2\] \[v = x_2z_1 - x_1z_2\] \[w = u^2z_1z_2 - v^3 -2v^2x_1z_2\] \[x_3 = 2uw\] \[y_3 = t(4v -w)-8y_1^2u^2\] \[z_3 = 8u^3\]
Wenn man zwei gleiche Punkte addiert, ändert sich die gerade genannte Formel zu:
\[t = Az_1^2 + 3x_1^2\] \[u = y_1z_1\] \[v = ux_1y_1\] \[w = t^2 - 8v\] \[x_3 = 2uw\] \[y_3 = t(4v-w)- 8y_1^2u^2\] \[z_3 = 8u^3\] mit dem Sonderfall von \(P1 = -P2\) folgt \(P1 + P2 = \infty\).
\subsection{Jacobi Darstellung}
Um einen noch größeren Speedup zu erreichen, kann man Jobische Koordinaten verwenden, da diese besonders effizient beim Verdoppeln sind. Hier ist die Darstellung (x,y,z) zu affin \((x/z^2, y/z^3)\) gewählt bei der Kurvenform \[y^2 = x^3 + Axz^4 + Bz^6\]Der Punkt unendlich ist definiert mit (1,1,0). 
\subsubsection{Addition}
Bei zwei unterschiedlichen Punkten wird wie folgt addiert:
\[r = x1z_2^2\] \[s = x_2z_1^2\] \[t = y1z_2^3\] \[v = s - r\] \[w = u - t\] \[x_3 = -v^3 - 2rv^2 + 2^2\] \[y_3 = -tv^3 + (rv^2-x_3)*w_1\] \[z_3 = vz_1z_2\]
Dadurch ergeben sich zwölf Multiplikationen und vier Quadrierungen.\\
Wenn man nun aber gleiche Punkte addiert, erkennt man direkt im Vergleich die Einfachheit der Formel:
\[v = 4x_1y_1^2\] \[w = 3x_1^2 + Az_1^4\] \[x_3 = -2v + w^2\] \[y_3 = -8y_1^4 + (v-x_3)w_1\] \[z_3 = 2y_1z_1\]. Wodurch man die Operationen auf drei Multiplikationen und sechs Quadrierungen reduziert. Das ist besonders schnell berechenbar für einen Computer.\\
Ein weiterer Vorteil dieser Darstellung ist die Kompatibilität zu affinen Punkten. Hier benötigt man für die Addition von einem Jacobi Punkt und einem affinen Punkt nur acht Multiplikationen und drei Quadrierungen.
\subsection{k-fache Multiplikation}
Wenn ein Punkt mit k multipliziert wird, unterscheidet man, dass bei \(k > 0\) \[P + P + ... + P = kP\] und bei \(k < 0\) \[ (-P) + (-P) + ... + (-P) = kP\] berechnet wird. Durch wiederholtes Verdoppeln erreicht man bei großen \(k\) die beste Performance und da die elliptischen Kurven auf endlichen Feldern basieren, muss man sich auch keine Gedanken um immer größere Koordinaten machen, da diese immer \( mod p\) gerechnet werden.