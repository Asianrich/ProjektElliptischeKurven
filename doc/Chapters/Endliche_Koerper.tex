\chapter{Endliche Körper}

\section{Primzahl-Generator}
In der Kryptographie arbeitet man mit Primzahlen, jedoch nicht mit kleinen Zahlen.
Jedoch bei höheren Primzahlen wird es schwieriger die Zahlen zu überprüfen, ob diese eine Primzahl ist.
Eiune davon ist der Fermat-Test.

\subsection{Fermat-Test}
Der Algorithmus läuft folgender maßen ab:\\
\\
1. Wähle eine Zahl $ a \in {2,3,4,...,n-2}$ \\
2. Überprüfe ob die Zahlen Teilerfremd sind. Ist diese, so ist die Zahl n keine Primzahl.\\
3. Falls die Zahlen teilerfremd sind, so führe folgende Formel aus.\\
$
a^{n-1} \equiv  1\ mod\ n
$\\
4. Ist dies erfüllt, dann ist die Zahl n vermutlich eine Primzahl. 
\\
Die Zahl a ist dementsprechend ein F-Zeuge, der aussagen kann, das die Zahl n eine Primzahl ist.
Aber es gibt Zahlen, welche F-Zeugen sind, jedoch ist die untersuchte Zahl keine Primzahl.
Anhand folgendem Beispiel können wir erkennen, wieso das ist:
\\
Testen wir die Zahl 15. So kommt bei der zufälligen Zahl 4 in der Rechnung: \\
$
4^{15-1}\ mod\ 15 = 1
$\\
Es wird angedeutet, das die Zahl 15 eine Primzahl ist. Welches es jedoch nicht ist. Da man durch $3 \cdot 5 = 15$ kommt.\\
Aus dem Grund ist es wichtig den Test mehrmals zu durchführen. Dadurch verringert sich die Fehlerwahrscheinlichkeit. Es erhöht aber die Laufzeit dadurch.\\

\subsection*{Algorithmus im Programm}
Es wird eine Zufällige Zahl generiert. Bei unseren Kryptographie-Beispielen wird mit einer 8192 Bit-Zahl erwartet. Daher geben wir auch die Länge der Bits mit hinzu.\\
Beim Fermat-Test wird eine 2.te Zahl (welcher kleiner ist) generiert und es wird nach dem ggt von den 2 zufälligen Zahlen ermittelt. Ist der ggT von den beiden größer als 1,\\
so ist die zu überprüfende Zahl keine Primzahl. Ist es 1, so geht es weiter mit dem Test. Wir nutzen die Formel oben und überprüfen, ob das Ergebnis 1 rausbringt. Ist dieser nicht 1, so ist es keine Primzahl. Ergibt es 1, so ist die 2.te generierte Zahl ein F-Zeuge und inkrementiert den Counter\\
Diesen Test für die 2.te Zahl generieren, wird beliebig mal ausgeführt und es wird überprüft, ob mehr F-Zeugen gibt als F-Lügner. Gibt es mehr F-Zeugen als F-Lügner, so ist unsere 1. generierte Zahl wahrscheinlich eine Primzahl.


\newpage
\section{Was ist ein endlicher Körper (Galois-Feld)?}

Ein Endlicher Körper oder auch Galoiskörper ist eine endliche Anzahl an Elementen, welche man die Grundoperationen Addition und Multiplikation anwenden kann.\\

Subtraktion und Division sind inverse Operationen. Diese Rechenarten sind jedoch nicht anders, wie man normal die Zahlen in $\mathbb{R}$, $\mathbb{Q}$.

Es ist eine Zyklische Gruppe $ \mathbb{Z}_p $ oder auch $ \mathbb{F}_p $. Hierbei muss p eine Primzahl sein. Damit alle Elemente darin enthalten sind. 

Allgemein: Die endlichen Körper $\mathbb{F}_p$ beinhalten p-Elemente. Diese werden durch die Ganzzahlen ${0,1,2,....,p-1}$ repräsentiert. 

Im Zusammenhang mit den Grundrechenoperationen wird es folgendermaßen definiert:

Addition: Wenn $ a, b \in \mathbb{Z}_p $, dann gilt $ a+b =r$ in $ \mathbb{Z}_p $, wobei $r \in [0, p-1]$ ist. Dies Zahl wird durch modulo p errechnet, nach der Addition.
Multiplikation: Wenn $ a, b \in \mathbb{Z}_p $, dann gilt $ a \cdot b =r$ in $ \mathbb{Z}_p $, wobei $r \in [0, p-1]$ ist. Dies Zahl wird durch modulo p errechnet, nach der Multiplikatioon.

Jetzt nur noch für die inverse Operationen definieren.

Additive Inverse (Subtraktion): Wenn $a \in \mathbb{Z}_p$, dann ist (-a) die additive Inverse von a in $\mathbb{Z}_p$. Dies ist auch die einzige Lösung zur Gleichung : $a + x \equiv 0$ (mod p)
Multiplikative Inverse (Division): Wenn $a \in \mathbb{Z}_p, a \neq 0 $,  dann ist $a^{-1}$ die multiplikative Inverse von a in $\mathbb{Z}_p$. Dies ist auch die einzige Lösung zur Gleichung : $a \cdot x \equiv 1$ (mod p)

Nehmen wir als Beispiel die Primzahl 2.

$ \mathbb{Z}_2 = {0,1} $

Hier hat man lediglich nur 2 Elemente. Rechnet man beispielsweise bei Addition alle Möglichkeiten, sieht es so aus: $0 + 0;\ 0 + 1 = 1;\ 1 + 1 = 0$ 
Um es einfacher und schöner zu gestalten erstellen wir eine Tabelle für Addition und Multiplikation.


\begin{table}[h]\caption{Addition und Multiplikation in $ \mathbb{Z}_2 = {0,1} $}
    \begin{tabular}{l|l|lll|l|ll}
    + & 0 & 1 &  & + & 0 & 1 &  \\ \cline{1-3} \cline{5-7}
    0 & 0 & 1 &  & 0 & 0 & 0 &  \\ \cline{1-3} \cline{5-7}
    1 & 1 & 2 &  & 1 & 0 & 1 & 
    \end{tabular}
\end{table}
\\

\begin{table}[h]\caption{Addition und Multiplikation in $ \mathbb{Z}_2 = {0,1} $}
    \begin{tabular}{l|l|l|lll|l|l|ll}
    + & 0 & 1 & 2 &  & $\cdot$ & 0 & 1 & 2 &  \\ \cline{1-4} \cline{6-9}
    0 & 0 & 1 & 2 &  & 0 & 0 & 0 & 0 &  \\ \cline{1-4} \cline{6-9}
    1 & 1 & 2 & 0 &  & 1 & 0 & 1 & 2 &  \\ \cline{1-4} \cline{6-9}
    2 & 2 & 0 & 1 &  & 2 & 0 & 2 & 1 & 
    \end{tabular}
\end{table}

\subsection*{Algorithmus im Programm}

Das Programm verläuft folgendermaßen. Zuerst erstellt man ein Objekt mit der gewählten Primzahl. Das Objekt besitzt alle möglichen Rechenoperationen in dem Umfeld.
Führt man eine Rechnung, wird überprüft ob die Zahl im jeweiligen Menge ist. Nach der Rechnung wird das Ergebnis mit modulo Prim ausgerechnet und zurückgeschickt. 

Die Rechenoperationen werden von der Modulare Arithmetik bereitgestellt und im Grunde werden die Werte weitergegeben.

\newpage
\section{Irreduzible Polynome}

Bevor man die Verbindung zu endlichen Körper macht, muss man die Irreduzible Polynome verstehen.\\
Irreduzible Polynome sind Polynome, welche man nicht mehr faktorisieren kann. Somit haben diese Polynome keine Nullstellen im $\mathbb{F}_{p} $

Beispiele für Irreduzible Polynome sind:\\
$
x^2+ x + 1\\
x^3 + x^2 + x + 1 \\
x^4 + x^3  + x^2 + x + 1\\
...
$
\\
\\
\subsection{Zusammenhang mit Endlichen Körper}

Der Körper K wird folgendermaßen aufgeschrieben : $ K := \mathbb{F}_{p^m}$. p für Primzahl und m für den Gradienten. Der Gradient sagt aus, welches Polynom genommen wird. 
Beispielsweise nimmt man für den Gradient 2 das Polynom $x^2 + x + 1$\\
Alle Berechnungen werden mit Polynomen ausgeführt und die X-Werte werden nicht ersetzt.
\\
Die Elemente die in $ K := \mathbb{F}_{p^m}$ repräsentieren sind:\\
\\
Allgemein: $  \mathbb{F}_{p^m} = \{p_{m-1} \cdot x^{m-1} + p_{m-2} \cdot x^{m-2} + ... + p_1 \cdot x + p_0 ; p_i \in \{0,1,..., p-1\} \}$\\
Man kann $\mathbb{F}_{p^m}$ auch so schreiben : $\mathbb{F}_{p^m}[x] = \frac{\mathbb{F}_{p}[x]}{Polynom} $\\
\\
Wie sehen dann die Polynom-Addition, und Multiplikation aus? Um es einfacher zu verstehen, nutzen wir einen Beispiel.
\\
Wir nehmen den Körper $ := \mathbb{F}_{2^2}$
\\
Hierbei wird unser Polynom $ x^2+ x + 1 $ benutzt. Da der Gradient = 2 ist. Wenn ein Polynom mit einem gleichen oder höheren Gradienten gibt, wird dieser durch Polynomdivision verkleinert und den Rest davon genommen.\\
bei $ x^2+ x + 1 $ können nur 4 Möglichkeiten geben. $ := \frac{\mathbb{F}_{2}[x]}{x^2+ x + 1} = \{0, 1, x, x+1\}$\\

Macht man eine Tabelle für diese Elemente, kommt folgendes dabei raus:

\begin{table}[h]\caption{TODO CAPTION}
    \begin{tabular}{l|l|l|l|lll|l|l|l|ll}
    +   & 0   & 1   & x   & x+1 &  & $\cdot$   & 0 & 1   & x   & x+1 &  \\ \cline{1-5} \cline{7-11}
    0   & 0   & 1   & x   & x+1 &  & 0   & 0 & 0   & 0   & 0   &  \\ \cline{1-5} \cline{7-11}
    1   & 1   & 0   & x+1 & x   &  & 1   & 0 & 1   & x   & x+1 &  \\ \cline{1-5} \cline{7-11}
    x   & x   & x+1 & 0   & 1   &  & x   & 0 & x   & x+1 & 1   &  \\ \cline{1-5} \cline{7-11}
    x+1 & x+1 & x   & 1   & 0   &  & x+1 & 0 & x+1 & 1   & x   & 
    \end{tabular}
\end{table}

\subsection*{Algorithmus im Programm}

Im Prinzip arbeitet das Programm, wie normal bei den Primzahl. Jedoch wird es mit Polynomen arbeiten, welche in ArrayList<BigInteger> abgespeichert wird.\\
Zuerst wird der Gradient bestimmt und es wird aus einer Liste paar Beispiel-Irreduzible Polynome genommen. Wenn ein ArrayList übergeben wird, bestimmt die Länge, den Höchsten Gradienten an.\\
Angenommen ist die Size der Liste q, so ist der Gradient q-1. Zuerst werden die Polynome mit den Grundrechenoperationen berechnet und anschließend über mod Irreduzibles Polynom abgerechnet. Anschließend wird der Rest ausgegeben.
\\

