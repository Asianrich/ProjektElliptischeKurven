\chapter{Endliche Körper}

\section{Primzahl-Generator}
In der Kryptographie arbeitet man mit Primzahlen, jedoch nicht mit kleinen Zahlen.\\
Aber woher weißt man, ob es sich um eine Primzahl handelt? Bei relativ kleinen Zahlen, kann man dies sehr schnell herausfinden.
Jedoch bei höheren Primzahlen wird es schwieriger die Zahlen zu überprüfen, ob es sich auch um eine Primzahl handelt.
Es gibt mehrere Primzahl-Test-Algorithmen, der bekannteste von allen ist dieser hier:

\begin{table}[!ht]
    \centering
        \begin{tabular}{l}
            \toprule
            \textbf{Algorithmus: Probedivision}\\
            \midrule
            Input: Eine Zufällige Zahl $n \in \mathbb{N} $\\
            Output: true = Primzahl, false = Keine Primzahl  \\
                                                       \\
                1. Beginne bei i = 2\\
                2. ist n ein vielfaches von i?\\
                    true: return false\\
                3. ist $ i > \sqrt{n}$?\\
                    false: inkrementiere i und gehe zu 2\\
                4. return true\\
           \bottomrule
        \end{tabular}
        \caption{Einfacher Primzahl-Test}
        \label{tab3}
    \end{table}

Man erkennt hier deutlich, das es bei kleinen Zahlen relativ zügig gehen kann.\\
Jedoch bei größeren Zahlen wird es eine ziemlich lange Zeit dauern. \\
\\
Mit diesen Gedanken, hat man nach Algorithmen gesucht, welche diesen einfachen Primzahl-Test ersetzen sollen.\\
\\Die bekanntestens sind:\\
\\
- Sieb des Eratosthenes\\
- Sieb von Atkin\\
- Probabilistische Primzahltests\\
- und viele weitere.\\
\\
Von all den aufgelisteten Tests, wird der Fermat-Test (probabilistischer Primzahltest) angeschaut.\\
\newpage
\subsection{Fermat-Test}

Der Fermat-Test beruht auf dem kleinen fermatischen Satz. Dieser lautet wie folgt:
\\
$
                    a^{n-1} \equiv  1\ mod\ n\\
$
\\
Zuerst schauen wir den Algorithmus an.\\
\begin{table}[!ht]
    \centering
        \begin{tabular}{l}
            \toprule
            \textbf{Algorithmus: Fermat-Test}\\
            \midrule
            Input: Eine zufällige Zahl $n \in \mathbb{N} $\\
            Output: true = Pseudoprimzahl, false = keine Primzahl  \\
                                                       \\
                                                       
            1. Wähle eine Zahl $ a \in {2,3,4,...,n-2}$ \\
            2. ggT berechnen. Überprüfe ob die Zahlen teilerfremd sind.\\ 
                -> Nicht teilerfremd => gemeinsamer Teiler => return false \\
            3. Falls die Zahlen teilerfremd sind, so führe folgende Formel aus:\\
                $
                    a^{n-1} \equiv  1\ mod\ n
                $\\
            4. Wiederhole für eine gewisse Anzahl an Versuchen.\\
                Merke Anzahl bestandener Versuche.\\
                
            5. F-Zeuge > F-Lügner?    
            -> true: return true \\
            -> false: return false\\
           \bottomrule
        \end{tabular}
        \caption{Fermat-Test}
        \label{tab3}
    \end{table}

Diese Lösungsmöglichkeit sieht zum ersten schön aus. Jedoch beinhaltet dieser Algorithmus einen Fehler.\\
Es gibt bestimmte Kombinationen, in der die untersuchte Zahl als Primzahl anerkannt wird, jedoch in Wahrheit keine Primzahl ist.\\

Anhand folgendem Beispiel können wir erkennen, wieso das ist:
\\
Es wird die Zahl 15 genommen und führt von 2-15 die Zahlen durch und listet es in einer Liste auf.\\
\newpage
\begin{table}[!ht]
    \centering
        \begin{tabular}{l}
            \toprule
            \textbf{Vielfache von 3}\\
            \midrule
            3, 6, 9, 12\\
           \bottomrule
        \end{tabular}
        \caption{Vielfache von 3}
        \label{tab3}
    \end{table}

    \begin{table}[!ht]
        \centering
            \begin{tabular}{l}
                \toprule
                \textbf{Vielfache von 5}\\
                \midrule
                5,10\\
               \bottomrule
            \end{tabular}
            \caption{Vielfache von 5}
            \label{tab3}
        \end{table}

        \begin{table}[!ht]
            \centering
                \begin{tabular}{l}
                    \toprule
                    \textbf{F-Zeuge}\\
                    \midrule
                    4, 11, 14\\
                   \bottomrule
                \end{tabular}
                \caption{F-Zeuge}
                \label{tab3}
            \end{table}

            \begin{table}[!ht]
                \centering
                    \begin{tabular}{l}
                        \toprule
                        \textbf{F-Lügner}\\
                        \midrule
                        2,7, 8, 13 \\
                       \bottomrule
                    \end{tabular}
                    \caption{F-Lügner}
                    \label{tab3}
                \end{table}

Würde man nur F-Zeugen bekommen, wird die Zahl 15 als Primzahl angedeutet. Die Zahl 15 ist aber keine Primzahl.
Aus dem Grund ist es wichtig den Test mehrmals durchzuführen. Dadurch verringert sich die Fehlerwahrscheinlichkeit, aber es erhöht sich die Laufzeit.\\

\subsection*{Algorithmus im Programm}

Um eine zufällige Zahl zu generieren, gibt man die Länge in Bits mit. In diesem Projekt wurde eine Primzahl, mit einer Bit-Länge von 8192 erwartet.

\begin{table}[!ht]
    \centering
        \begin{tabular}{l}
            \toprule
            \textbf{Algorithmus: Primzahl-Generator}\\
            \midrule
            Input: Länge der Primzahl\\
            Output: Primzahl  \\
            \\
            1. Zahl generiert\\
            2. Fermat-Test anwenden, mit einer gewissen Anzahl an Versuchen\\
                false: beginne bei 1. \\
            3. return Primzahl\\
           \bottomrule
        \end{tabular}
        \caption{Primzahl-Generator}
        \label{tab3}
    \end{table}

\newpage
Führt man den Algorithmus aus, so kommt man eventuell auf die Primzahl :

70 6815 0405
5962 8624 0116 5582 8814
1890 4999 7822 5925 9076
9503 5507 1619 2423 0111
6620 5520 8288 3610 7731
9769 4969 0297 4888 4070
3257 5833 8064 6528 7972
3291 7661 0446 8045 1640
0051 6702 9150 4361 6802
8389 3226 0098 4356 6161
1330 4262 7433 1919 7858
0365 6198 8262 0250 9889
0007 0954 7574 9383 1886
5361 3190 7877 2205 3670
4087 2171 7091 6374 3838
8716 3932 5296 3638 2140
4949 1393 3782 5072 4723
8804 6806 7614 3346 5804
5628 0683 2443 5101 6267
9171 2074 5458 0355 7929
7744 9444 1421 5092 6350
1162 8496 6988 9236 0338
3073 1279 6721 0801 9128
7263 5150 8342 4633 7071
8726 3737 9368 5065 8952
9713 4329 7856 0992 6224
2713 6350 4753 0207 8413
7888 8748 0655 9121 6942
9384 5841 1784 2617 6752
2643 0070 3556 3276 6686
6523 6072 2767 0478 8204
3629 2550 8690 0935 0632
7571 2249 8673 5817 9463
3439 3486 6162 1529 9621
3988 9362 9652 6810 5885
7633 5373 9720 0378 2639
0125 7017 6324 9307 8768
5224 1751 1885 3081 3584
1722 1685 4433 3817 4480
6016 5101 8674 2180 2737
9668 4275 7149 1326 9800
4087 1250 2661 4749 0951
7078 6570 0339 9965 4623
0540 0957 6548 0165 1284
4902 1116 0441 2929 0051
0857 9995 9240 9802 0822
9474 7571 0623 7772 5932
1672 7224 4630 0980 9029
2147 5787 7331 0174 1858
9750 2773 6215 9885 3344
8006 3253 0604 6003 4384
2821 4673 2124 8686 8459
9423 0273 6968 4661 7844
5424 0255 2547 4792 4467
5374 9401 1206 7899 9391
0968 8939 7038 7423 6856
6727 7350 0374 2514 6856
8184 5213 8164 4258 9305
5193 7144 0916 7515 0860
7244 0174 0141 8084 6042
5181 0634 1126 9636 4500
2538 5977 1624 4032 2865
6823 7439 8977 1815 6120
7864 1853 6229 5787 0600
4382 5331 3365 1603 5651
6751 3330 6865 9910 0308
0063 4913 2569 2019 3094
4494 1725 4427 0941 9613
7271 6810 4820 9910 8363
3502 9174 0854 8677 7734
7568 0720 4455 8979 7970
5022 1557 6524 0814 1278
0843 8718 2630 3908 1154
0097 0824 7859 4337 5885
7050 0393 6496 8972 4092
1182 0078 5656 0493 9605
3816 6074 4358 4332 6214
8690 7009 6825 5029 8782
7415 1540 5975 1381 2588
1923 3731 2939 7576 0031
7781 9263 9960 8730 9804
0411 6651 5594 1936 1198
8630 9016 9448 2244 9085
7228 5217 1963 4461 0055
0330 2220 6229 6277 4812
8053 7639 0871 8791 4674
2159 7828 9380 7599 1756
9444 0479 9931 8911 5307
7384 0105 3550 1120 6437
2188 8508 6491 4124 7819
0968 8775 3958 5616 3999
7092 4953 6260 8121 3276
9665 6900 9134 2768 6308
8067 1280 5023 8310 4702
1587 4989 6780 0320 7794
3957 4394 0139 7926 4275
5192 6627 4494 8587 9184
5545 5525 6075 3829 0180
0868 1184 4304 9256 9086
0751 5701 2096 2199 9694
2375 1322 0102 3865 9208
3656 9461 5931 3256 9318
8649 6485 2471 1023 7452
6270 1542 3029 3085 6406
4691 0217 4783 4265 3842
1920 4077 0900 8805 7200
6880 2156 9270 3522 4746
6514 7127 1080 7195 0979
9595 9529 3246 9726 3472
4012 0939 3505 3249 8908
8835 0540 0900 3795 9129
0667 8220 6335 1467 9553
0848 7085 5458 0141 1058
7791 3737 2972 0536 9327
7136 3663 3385 1034 2973
6790 2383 6961 9534 6174
4388 2640 5243 9154 9655
7946 0048 8789 7953 8528
3864 1039 7464 3673 5184
1846 6819 6396 5498 5830
4678 7511 1233 6299 5444
8304 9448 5627 1167 1761
3370 4359 6385 2667 9925
6225 6901 0792 6579 
\\
\\
Diese Zahl wurde von anderen Programmen wie \url{https://www.alpertron.com.ar/ECM.HTM} nachgeprüft. Weitere Zahlen findet man im Anhang\\
\\
\subsection*{Fazit zum Fermat-Test}

Der Fermat-Test läuft um einiges schneller als die Probedivision.\\
Jedoch sagt dieser Test nicht aus, ob es sich tatsächlich um eine Primzahl handelt.\\
Je mehr Versuche man macht, desto geringer wird die Fehlerwahrscheinlichkeit. \\
Will man diese Fehlerwahrscheinlichkeit verringern, wird die Laufzeit des Programmes, sich deutlich erhöhen.  \\
\\
Bei der Programmierung gab es keine Schwierigkeiten, da die Formeln und das Abschlusskriterium schon angegeben waren.\\
Nur dauerte das Generieren der Primzahl relativ lang bei 8192-Bitlänge. Wenn man in 2er-Potenzen hochgehen würde, so hat sich die Laufzeit exponentiell erhöht.\\
Würde man jeden einzelnen Schritt debuggen, so hätte man bemerkt, das die Potenzrechnungen (bei hohen Zahlen) einige Minuten dauert.
Dies kann daran liegen, dass die Java-Bibliothek BigInteger schlecht für diskrete Mathematik optimiert ist oder der Rechner nicht so schnell rechnen konnte.
Als man eine mögliche Zahl bekam, wurde dies auf der oben genannten URL getestet. Selbst die Seite hatte einige Minuten gebraucht, um die Zahl zu überprüfen.\\ 
\newpage
\section{Was ist ein endlicher Körper (Galois-Feld)?}

Ein endlicher Körper oder Galois-Körper ist ein algebraisches System, welches eine endliche Menge mit definierten Addition und Multiplikation.\\
Der Endliche Körper hat folgende Eigenschaften:


\begin{table}[!ht]
    \centering
        \begin{tabular}{l}
            \toprule
            \textbf{Eigenschaften der Endlichen Körper}\\
            \midrule
            - Die Endliche Menge ist eine abelsche Gruppe, in der \\
            die Rechennoperation Addition berechnet werden kann.\\
            - Endliche Menge ohne dem Nullelement ist eine abelsche Gruppe, \\
            in der die Multiplikation stattfinden kann.\\
            - Endliche Körper sind auch zyklische Gruppen mit der Schreibweise: $ \mathbb{Z}_p $. \\
           \bottomrule
        \end{tabular}
        \caption{Beispiele für Irreduzible Polynome}
        \label{tab3}
    \end{table}

Der Unterschied zwischen x-Beliebige Zyklische Gruppen und einem Endlichen Körper ist, das der Endliche Körper durch einer Primpotenz $ p^n $ beschrieben werden kann.

Die Anzahl an Elementen in einem endlichen Körper nennt man Ordnung des Endlichen Körper. Die Ordnung des Endlichen Körper kann nur als Primpotenz dargestellt werden.\\

Allgemein: Die endlichen Körper $\mathbb{F}_{p^n}$ beinhalten p-Elemente und n. Diese werden durch die Ganzzahlen ${0,1,2,....,p-1}$ repräsentiert. 
\\

Wie ist das nun zu verstehen? Schauen wir den normalen Zyklischen Ring $ \mathbb{Z}$ an.

\begin{table}[!ht]
    \begin{tabular}{lllllll}
    ... & -6 & -5 & -4 & -3 & -2 & 1  \\
    0  & 1  & 2  & 3  & 4  & 5  & 6  \\
    7  & 8  & 9  & 10 & 11 & 12 & 13 \\
    14 & 15 & 16 & 17 & 18 & 19 & ...

    \end{tabular}
    \caption{Zahlen in $\mathbb{Z}$}
\end{table}

Jetzt rechnen wir die Zahlen in $\mathbb{Z}$ zu $\mathbb{Z}_p$ mit p = 7.\\

\begin{table}[!ht]
    \begin{tabular}{lllllll}
    ...  & 1  & 2  & 3  & 4  & 5  & 6  \\
    0  & 1  & 2  & 3  & 4  & 5  & 6  \\
    0  & 1  & 2  & 3  & 4  & 5  & 6 \\
    0  & 1  & 2  & 3  & 4  & 5  & ...

    \end{tabular}
    \caption{Zahlen in $\mathbb{Z}/\mathbb{Z}_7$}
\end{table}

Alle Zahlen in $\mathbb{Z}$ geben nun die Restwerte aus.\\
Jetzt stellt sich nun die Frage, ob man über diesen Körper Rechenoperationen durchführen kann.
Dazu müssen die Rechenoperationen definiert werden.

\begin{table}[!ht]
    \centering
        \begin{tabular}{l}
            \toprule
            \textbf{Addition in der Endlichen Körper}\\
            \midrule
            Wenn $ a, b \in \mathbb{Z}_p $, dann gilt $ a+b =r$ in $ \mathbb{Z}_p $, wobei $r \in [0, p-1]$ ist.\\
           \bottomrule
        \end{tabular}
        \caption{Addition in der Endlichen Körper\cite{sec1}}
        \label{tab3}
    \end{table}

    \begin{table}[!ht]
        \centering
            \begin{tabular}{l}
                \toprule
                \textbf{Multiplikation in der Endlichen Körper}\\
                \midrule
                Wenn $ a, b \in \mathbb{Z}_p $, dann gilt $ a \cdot b =r$ in $ \mathbb{Z}_p $, wobei $r \in [0, p-1]$ ist.\\
               \bottomrule
            \end{tabular}
            \caption{Multiplikation in der Endlichen Körper\cite{sec1}}
            \label{tab3}
        \end{table}

Subtraktion und Division sind inverse Operationen. Diese Rechenarten sind jedoch nicht anders, wie man normal die Zahlen in $\mathbb{R}$, $\mathbb{Q}$.

Diese sind wie folgt definiert:
\begin{table}[!ht]
    \centering
        \begin{tabular}{l}
            \toprule
            \textbf{Additive Inverse (Subtraktion) in Endlichen Körpern}\\
            \midrule
            Wenn $a \in \mathbb{Z}_p$, dann ist (-a) die additive Inverse von a in $\mathbb{Z}_p$.\\
             Dies ist auch die einzige Lösung zur Gleichung : $a + x \equiv 0$ (mod p)\\
                        \bottomrule
        \end{tabular}
        \caption{Additive Inverse (Subtraktion) in Endlichen Körpern \cite{sec1}}
        \label{tab3}
    \end{table}

    \begin{table}[!ht]
        \centering
            \begin{tabular}{l}
                \toprule
                \textbf{Multiplikative Inverse (Division) in Endlichen Körpern}\\
                \midrule
                Wenn $a \in \mathbb{Z}_p, a \neq 0 $,  dann ist $a^{-1}$ das multiplikative Inverse von a in $\mathbb{Z}_p$.\\
                 Dies ist auch die einzige Lösung zur Gleichung : $a \cdot x \equiv 1$ (mod p)          \\
                 \bottomrule
            \end{tabular}
            \caption{Multiplikative Inverse (Division) in Endlichen Körpern \cite{sec1}}
            \label{tab3}
        \end{table}
\newpage
\subsection*{Beispiel}
Nehmen wir als Beispiel die Primzahl 2.

$ \mathbb{Z}_2 = {0,1} $

Hier hat man lediglich nur 2 Elemente. Berechnet man beispielsweise bei der Addition alle Möglichkeiten, sieht das so aus: $0 + 0;\ 0 + 1 = 1;\ 1 + 1 = 0$ 
Um es einfacher und schöner zu gestalten, erstellen wir eine Tabelle für Addition und Multiplikation.


\begin{table}[!ht]\caption{Addition und Multiplikation in $ \mathbb{Z}_2 = {0,1} $}
    \begin{tabular}{l|l|lll|l|ll}
    + & 0 & 1 &  & + & 0 & 1 &  \\ \cline{1-3} \cline{5-7}
    0 & 0 & 1 &  & 0 & 0 & 0 &  \\ \cline{1-3} \cline{5-7}
    1 & 1 & 2 &  & 1 & 0 & 1 & 
    \end{tabular}
\end{table}
\begin{table}[!ht]\caption{Addition und Multiplikation in $ \mathbb{Z}_2 = {0,1} $}
    \begin{tabular}{l|l|l|lll|l|l|ll}
    + & 0 & 1 & 2 &  & $\cdot$ & 0 & 1 & 2 &  \\ \cline{1-4} \cline{6-9}
    0 & 0 & 1 & 2 &  & 0 & 0 & 0 & 0 &  \\ \cline{1-4} \cline{6-9}
    1 & 1 & 2 & 0 &  & 1 & 0 & 1 & 2 &  \\ \cline{1-4} \cline{6-9}
    2 & 2 & 0 & 1 &  & 2 & 0 & 2 & 1 & 
    \end{tabular}
\end{table}
\newpage
\subsection*{Algorithmus im Programm}

Das Programm verläuft folgendermaßen: Zuerst erstellt man ein Objekt mit der gewählten Primzahl. Das Objekt besitzt alle möglichen Rechenoperationen in dem Umfeld.
Führt man eine Rechnung aus, so wird überprüft, ob die Zahl in der jeweiligen Menge enthalten ist. Nach der Rechnung wird das Ergebnis mit modulo Prim ausgerechnet und zurückgegeben. 

Die Rechenoperationen werden von der modularen Arithmetik bereitgestellt und im Grunde werden die Werte weitergegeben.
\\
In der Programmierung gibt es keine Probleme, da die Modulare Arithmetik schon fertig implementiert wurde. 
Lediglich musste überprüft werden, ob die angegebene Zahl eine Primzahl ist oder nicht. Da aber der Fermat-Test schon fertig war, konnte man den Test einfach aufrufen. 
\\
\newpage
\section{Primzahlpotenz}

\subsection*{Irreduzibles Polynom}


Bevor man die Verbindung zu endlichen Körpern macht, muss man die irreduziblen Polynome verstehen.\\
Damit Polynome irreduzibel sind, müssen die Irreduziblitätskriterien erfüllt sein.\\

\begin{table}[!ht]
    \centering
        \begin{tabular}{l}
            \toprule
            \textbf{Lemma}\\
            \midrule
            Sei K ein Körper und $ f \in K[T]$. Dann gilt\\
            a) Jedes Polynom $ f \in K[T]$ vom Grade 1 ist irreduzibel.\\
            b) Ein Polynom $ f \in K[T]$ vom Grade $\geq 2$, das eine Nullstelle in K besitzt,\\
             ist reduzibel \\
            c) $grad(f) \in {2,3}$: f irreduzibel über K $ \leftrightarrow $ \\
            d) Besitzt ein Polynom $ $\\
           \bottomrule
        \end{tabular}
        \caption{Beispiele für irreduzible Polynome}
        \label{tab3}
    \end{table}


Kurz zusammengefasst: Irreduzible Polynome sind Polynome, welche man nicht mehr faktorisieren kann. Somit haben diese Polynome keine Nullstellen im $\mathbb{F}_{p} $


\begin{table}[!ht]
    \centering
        \begin{tabular}{l}
            \toprule
            \textbf{Beispiele für Irreduzible Polynome}\\
            \midrule
            $ x^2+ x + 1$\\
            $x^3 + x^2 + x + 1$ \\
            $x^4 + x^3  + x^2 + x + 1$\\
            $...$\\
           \bottomrule
        \end{tabular}
        \caption{Beispiele für Irreduzible Polynome}
        \label{tab3}
    \end{table}

\subsection{Zusammenhang mit endlichen Körpern}

Der Körper K wird folgendermaßen aufgeschrieben $ K := \mathbb{F}_{p^m}$. p für Primzahl und m für den Gradienten. Der Gradient sagt aus, welches Polynom genommen wird. 
Beispielsweise nimmt man für den Gradienten 2 das Polynom $x^2 + x + 1$\\
Alle Berechnungen werden mit Polynomen ausgeführt und die X-Werte werden nicht ersetzt.
\\
Die Elemente die in $ K := \mathbb{F}_{p^m}$ repräsentieren sind:\\
\\
Allgemein: $  \mathbb{F}_{p^m} = \{p_{m-1} \cdot x^{m-1} + p_{m-2} \cdot x^{m-2} + ... + p_1 \cdot x + p_0 ; p_i \in \{0,1,..., p-1\} \}$\\
Man kann $\mathbb{F}_{p^m}$ auch so schreiben : $\mathbb{F}_{p^m}[x] = \frac{\mathbb{F}_{p}[x]}{Polynom} $\\
\\
Wie sieht dann die Polynom-Addition und Multiplikation aus? Um es einfacher zu verstehen, nutzen wir ein Beispiel.
\\
Wir nehmen den Körper $ := \mathbb{F}_{2^2}$
\\
Hierbei wird unser Polynom $ x^2+ x + 1 $ benutzt, da der Gradient = 2 ist. Wenn es ein Polynom mit einem gleichen oder höheren Gradienten gibt, wird dieses durch Polynomdivision verkleinert und der Rest davon genommen.\\
Bei $ x^2+ x + 1 $ kann es nur 4 Möglichkeiten geben. $ := \frac{\mathbb{F}_{2}[x]}{x^2+ x + 1} = \{0, 1, x, x+1\}$\\

Macht man eine Tabelle für diese Elemente, kommt folgendes dabei heraus:

\begin{table}[h]\caption{Tabellen Grundrechenoperationen}
    \begin{tabular}{l|l|l|l|lll|l|l|l|ll}
    +   & 0   & 1   & x   & x+1 &  & $\cdot$   & 0 & 1   & x   & x+1 &  \\ \cline{1-5} \cline{7-11}
    0   & 0   & 1   & x   & x+1 &  & 0   & 0 & 0   & 0   & 0   &  \\ \cline{1-5} \cline{7-11}
    1   & 1   & 0   & x+1 & x   &  & 1   & 0 & 1   & x   & x+1 &  \\ \cline{1-5} \cline{7-11}
    x   & x   & x+1 & 0   & 1   &  & x   & 0 & x   & x+1 & 1   &  \\ \cline{1-5} \cline{7-11}
    x+1 & x+1 & x   & 1   & 0   &  & x+1 & 0 & x+1 & 1   & x   & 
    \end{tabular}
\end{table}
\subsection*{Beispiel}

Wir betrachten  $\mathbb{Z}_{2^3}[x]$:\\

\begin{table}[!ht]
    \begin{tabular}{lllll}
    0             & 1         & $x$       & $x+1$       & $x^2$       \\
    $x^2+1$       & $x^2+x$   & $x^2+x+1$ & $x^3$       & $x^3+1$     \\
    $x^3+x$       & $x^3+x+1$ & $x^3+x^2$ & $x^3+x^2+1$ & $x^3+x^2+x$ \\
    $x^3+x^2+x+1$ & $x^4$     & $x^4+1$   & $x^4+x$     & $x^4+x+1$   \\
    ...           &           &           &             &            
    \end{tabular}
    \caption{Mögliche Polynome über $\mathbb{Z}_2 [x]$}
    \end{table}
Nehmen wir als irreduzibles Polynom $ x^3 + x + 1$, dann gibt es folgende Tabelle raus:\\
\begin{table}[!ht]
    \begin{tabular}{lllll}
    0       & 1       & $x$       & $x+1$   & $x^2$   \\
    $x^2+1$ & $x^2+x$ & $x^2+x+1$ & $x+1$   & $x$     \\
    1       & 0       & $x^2+x+1$ & $x^2+x$ & $x^2+1$ \\
    $x^2$   & $x^2+x$ & $x^2+x+1$ & $x^2$   & $x^2+1$ \\
    ...     &         &           &         &        
    \end{tabular}
    \caption{$\mathbb{Z}_{2^3}[x]$}
    \end{table}


    Additions Beispiel:\\
    \\
    $ (x^4+x+1) + (x^4+x^3+x) = (x^3+1)\ mod (x^3 + x + 1) = x $\\
    \\

    Multiplikations Beispiel:\\
    \\
    $
    (x^2+1) \cdot (x^2+x+1) = (x^4+x^3+x+1) \ mod (x^3 + x + 1) = x^2 + x\\
    $
    \\
\subsection*{Algorithmus im Programm}

Im Prinzip arbeitet das Programm, wie normal bei einer Primzahl. Jedoch wird es mit Polynomen arbeiten, welche in einer ArrayList<BigInteger> abgespeichert werden.\\
Zuerst wird der Gradient bestimmt und es wird aus einer Liste ein Beispiel-Polynom genommen. Wenn eine ArrayList übergeben wird, bestimmt die Länge den höchsten Gradienten.\\
Angenommen die Länge der Liste sei q, so ist der Gradient q-1. Zuerst werden die Polynome mit den Grundrechenoperationen berechnet und anschließend wird mit einem irreduziblen Polynom der Restwert bestimmt.
\\

