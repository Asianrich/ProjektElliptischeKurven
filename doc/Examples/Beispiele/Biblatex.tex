%*********************************************
%*	Biblatex-Beispiel
%*********************************************
\section{Beispiel für BibLaTeX}

BibLaTeX ist ein Package, das einem die Arbeit mit Zitaten bzw. Quellenangaben erleichtern kann. Mit JabRef (\autoref{sec:Werkzeuge}) ist es möglich
\textit{*.bib}-Dateien zu erstellen, in denen alle Angaben zu Autor, Buchtitel, Erscheinungsdatum usw. hinterlegt werden, welche zum passenden Zeitpunkt
abgerufen werden können. Das Literaturverzeichnis wird mittels \lstinline{\printbibliography} ausgegeben.

Im Allgemeinen wird im Literaturverzeichnis auch nur jene Literatur aufgenommen, die auch in der \textit{*.tex}-Datei referenziert wird. Danach ist es wichtig
nicht nur mit \textit{Pdf\-LaTeX}, sondern auch mit \textit{BibLaTeX} zu kompilieren, damit die zitierten Einträge in die verschiedenen Hilfsdateien aufgenommen
werden können. %Hinweis: Pdf\-LaTeX teilt LaTeX mit, dass nur zwischen Pdf und LaTeX getrennt werden darf


\subsection*{Einige Zitate}
In diesem Satz könnten wir auf \cite{knuth:1976} verweisen, ebenso auf das wichtige Werk \cite{dueck:trio}. Wenn uns das nicht genug ist, sollten wir das anmerken,
was in \cite{sommerville:1992} geschrieben wurde. Im Zweifelsfall verweisen wir auf eine einzelne Seite, wie in \cite[112]{bentley:1999} zu finden. 

Üblicherweise wird auch der Name des Autors bzw. der Autoren genannt, also beispielsweise bei einem Verweis auf \citeauthor{knuth:1976} \cite{knuth:1976} oder 
auch bei mehreren Autoren \citeauthor{cormen:2001} \cite{cormen:2001}. LaTeX stellt Mechanismen zur Verfügung, auch dies automatisiert zu erledigen.




